% !TeX root = main.tex
\section{Existence of Solutions}
\writenumberedtocentry{chapter}{Existence of Solutions}{\the\sectcount}

In this lecture, we discuss inverse function theorem on Banach spaces, Fredholm alternative on Hilbert spaces and Sobolev embedding.
These utils enable us to discuss the existence of solutions.

First we define the Frechet derivative.
Let $B_1,B_2$ be two Banach spaces, $U_1\subset B_1,U_2\subset B_2$ are open subsets.
\medskip

\advance\propcount by 1
\noindent{\bf Definition~\propnumber}.
A map $F:B_1\to B_2$ is called {\it Frechet differentiable} at $u\in U_1$ if there exists a bounded linear mapping $L:B_1\to B_2$ such that
$${\|F(u+h)-F(u)-Lh\|_{B_2}\over\|h\|_{B_1}}\to 0\ {\rm as}\ h\to 0\ {\rm in}\ B_1.$$
The linear mapping $L=:\delta_uF$ is called the {\it Frechet derivative} of $F$ at $u$.
We say $F$ is a $C^1$-map (continuously Frechet differentiable) at $u$ if the map $v\to\delta_vF$ is continuous at $u$.
\medskip

Now we state the inverse function theorem and implicit function theorem.
\advance\propcount by 1
\proclaim Theorem~\propnumber.
\definexref{thm:inverse}{\propnumber}{thm}
Let $\Phi:U\to V$ be a $C^1$-map.
Assume $\Phi(0)=0$, $\delta_u\Phi(0)$ is an isomorphism, then there exists an open neighborhood $U_0$ of $0$ and $V_0\subset B_2$ such that for all $v\in V_0$, $\Phi(u)=v$ is solvable for $u$.

\advance\propcount by 1
\proclaim Theorem~\propnumber.
\definexref{thm:implicit}{\propnumber}{thm}
Let $B_1,B_2$ be Banach spaces, $W\subset B_1\times X$ be an open subset.
On open subset $V\subset B_2$, $\Phi:W\to V$ is $C^1$, $\Phi(u_0,t_0)=0$, $u_0\in B_1$, $t_0\in X$.
Assume the bounded linear operator $\delta_u\Phi(u_0,t_0)$ is invertible, then there exists a small neighborhood $V_0$ of $t_0$ such that $\Phi(u,t)=0$ is solvable for $u=u(t)$.

\advance\propcount by 1
\noindent{\it Remark~\propnumber}.
Apply to PDE, consider $F(u)=v$, $F$ is $C^1$ between Banach spaces (e.g.\ $C^{0,\alpha},W^{k,p},L^p$) if the linearized operator $\delta_uF(u_0)$ is invertible.
\medskip

\advance\propcount by 1
\noindent{\bf Example~\propnumber}.
For an elliptic operator $F$, if $\delta F$ is uniformly elliptic, implicit function theorem implies $F(u)=v$ is locally solvable for ``weak solution'' in $W^{k,p}$ sense.
Next, use the elliptic estimate (Schauder estimate or $L^p$-estimate) to boost up regularity to get regular enough solution.
\medskip

\demo
Take $U=V$, assume $dF(0)={\rm id}$ (otherwise, consider $\tilde{F}=(dF^{-1})(0)F$).
Let $F(u)=u+\eta(u)$, then
$$F(u)=v\iff u=v-\eta(u).$$
Let $G(u)=v-\eta(u)$, then the equation is equivalent to $G(u)=u$.
Note $dF(u)=Iu+d\eta(u)$, hence $dF(0)=I+d\eta(0)$, so $d\eta(0)=0$.
Also $d\eta$ is a bounded linear operator, hence there exists a $\delta>0$, such that for $\|u\|<\delta$, we have
$$\|d\eta(u)-d\eta(0)\|<{1\over2}.$$
Thus for $u_1,u_2$ with $\|u_i\|<\delta$, we have
$$\|\eta(u_1)-\eta(u_2)\|\leq{1\over2}\|u_1-u_2\|.$$
Now let $u_1=v$, $u_2=G(u_1)$,\dots,$u_n=G(u_{n-1})$,\dots
We thus have
$$\eqalign{
	\|u_{n+1}-u_{m+1}\|&=\|G(u_n)-G(u_m)\|\cr
	&\leq{1\over2}\|u_n-u_m\|\cr
	&\leq\left({1\over2}\right)^{n-m-1}\|u_2-u_1\|,
}$$
hence $\{u_n\}$ is a Cauchy sequence.
Therefore $\lim_{n\to\infty}u_n=u_\infty$ exists in Banach space, and $G(u_n)=u_{n+1}$ implies $G(u_\infty)=u_\infty$.
\enddemo

Next we discuss Arzela--Ascoli theorem.
\advance\propcount by 1
\proclaim Theorem~\propnumber.
Let $X$ be a compact metric space, $\{f_k(x)\}$ be a sequence of functions such that
\smallskip
\item {(1)} $f_k(x)$ is uniformly bounded, i.e.\ there exists $M$ such that $\forall x\in X$, $\forall k\in\mathbb{N}$, we have $|f_k(x)|<M$;
\item {(2)} $f_k(x)$ is uniformly equicontinuous, i.e.\ for any $\varepsilon>0$, there exists $\delta>0$, such that for $\|x-y\|<\delta$, $|f_k(x)-f_k(y)|<\varepsilon$ independent from $x$ and $k$.
\smallskip
Then there exists a subsequence $\{f_{k_j}\}$ such that $f_{k_j}${\raise .5ex\hbox{$\to$}}{\kern -1em\lower .5ex\hbox{$\to$}}$f$ in some compact subset of $X$.
\medskip

\advance\propcount by 1
\noindent{\it Remark~\propnumber}.
For PDE, $\{f_k\}\subset W^{k,p},L^p$ or $C^{0,\alpha}$.
Then (1) requires checking the boundedness of Sobolev, H\"older norms, and (2) requires checking the boundedness of weak derivatives.
\medskip

Now we discuss Fredholm alternative.
For simplicity, we only discuss Hilbert space version.
\medskip

\advance\propcount by 1
\noindent{\bf Definition~\propnumber}.
Let $V_1,V_2$ be normed linear spaces, $T:V_1\to V_2$ is called a {\it compact operator} if for any bounded sequence $\{u_k\}\subset V_1$, there exists a subsequence $\{u_{k_j}\}$ such that $\{Tu_{k_j}\}$ converges in $V_2$.
\medskip

Assume $H$ is a Hilbert space, $T:H\to H$ is a compact operator, then one can check that $T^*:H\to H$ is also a compact operator.
\advance\propcount by 1
\proclaim Theorem~\propnumber.
Assumption as above, we have
\smallskip
\item {(1)} The spectral of $T$ is discrete, i.e.\ $\Lambda=\{\lambda_i\in\mathbb{R}|\lambda_i\neq 0,Tx=\lambda_ix\}$ is discrete.
Moreover, $\ker(\lambda_iI-T)$ and $\ker(\lambda_iI-T^*)$ are of finite dimensional.
\item {(2)} If $\lambda\neq 0$, $\lambda\in\Lambda$, then
$$(\lambda I-T)x=y\quad{\rm and}\quad(\lambda I-T^*)x=y$$
have unique solutions $x$ for each $y$ and $(\lambda I-T)^{-1},(\lambda I-T^*)^{-1}$ exist and are bounded.
\item {(3)} For $\lambda\in\Lambda$, $(\lambda I-T)x=y$ is solvable if and only if $y\perp\ker(\lambda I-T^*)$, $(\lambda I-T^*)x=y$ is solvable if and only if $y\perp\ker(\lambda I-T)$.

\advance\propcount by 1
\noindent{\it Remark~\propnumber}.
We often use (Hodge) Laplacian $\Delta$ on a Riemannian manifold and $\Delta_{\bar\partial}$ on a complex manifold.
\medskip

\noindent{\bf Application}: Prescribed Gaussian curvature equation on a compact Riemannian surface.

Assume $(\Sigma,ds_0^2)$ is a compact Riemannian surface, $K_0$ is the Gaussian curvature of $ds_0^2$.
Let $ds^2=e^{2f}ds_0^2$ be the hyperbolic metric $K=-1$.
Then $f$ satisfies
$$\Delta f+e^{2f}=-K_0,\eqdef{eq:curv eq 1}$$
where $\Delta$ is the Hodge Laplacian.
We rewrite~\eqref{eq:curv eq 1}~as $\Delta f+e^{2f}-1=-K_0-1$.
Let $F(f)=\Delta f+e^{2f}-1$, then $\delta F(\varphi)=(\Delta+2)\varphi$.
Since all eigenvalues of $\Delta>0$, $\Delta+2$ is invertible.
We apply iteration on
$$(\Delta+2)f=(-K_0-1)+(1+2f-e^{2f}).\eqdef{eq:curv eq 2}$$
Set
$$\eqalign{
	&f_1=(\Delta+2)^{-1}(-K_0-1)\cr
	&f_2=f_1+(\Delta+2)^{-1}(1+2f_1-e^{2f_1})\cr
	&\cdots\cr
	&f_n=f_{n-1}+(\Delta+2)^{-1}(1+2f_{n-1}-e^{2f_{n-1}})\cr
	&\cdots
}$$
We can check when $\|-K_0-1\|<\varepsilon$ ($\varepsilon=1/2$ will do) in some Sobolev norm, \eqref{eq:curv eq 2}~has unique solution $f$ for $\|f\|<\delta$.
\enddemo

\advance\propcount by 1
\noindent{\bf Definition~\propnumber}.
A bounded linear operator $T:H\to H$ is called {\it Fredholm} if
\smallskip
\item {(1)} Both $\ker(T),\mathop{\rm coker}(T)$ are of finite dimensional.
\item {(2)} $\mathop{\rm im}(T)$ is closed.
\item {(3)} $\mathop{\rm Index}(T)=\dim\ker(T)-\dim\mathop{\rm coker}(T)$.
\medskip

Now we review some aspects about Sobolev spaces.
Let $\Omega\subset\mathbb{R}^n$ be a bounded domain, define {\it Sobolev space}
$$W^{k,p}(\Omega):=\left\{f\in L^p(\Omega)\left|\int_\Omega|\nabla^if|^p<+\infty,\ i=1,\cdots,k\right.\right\}\subset L^p(\Omega).$$
We define {\it Sobolev norm}
$$\|f\|^p_{W^{k,p}(\Omega)}:=\sum_{|\alpha|\leq k}\int_\Omega|\nabla^\alpha f|^p.$$
Define $W^{k,p}_0(\Omega)$ to be the completion of $C^k_0(\Omega)$ with respective to $W^{k,p}$-norm, whose elements should be understood as Cauchy sequences.

Now we state the Sobolev embedding theorem.
\advance\propcount by 1
\proclaim Theorem~\propnumber.
There are embeddings
$$W^{k,p}_0(\Omega)\hookrightarrow\left\{\eqalign{
	&W^{l,q}_0(\Omega)\ {\sl if}\ 1\leq q\leq p,0\leq l\leq k,l-{n\over q}<k-{n\over p}<l;\cr
	&L^{np\over n-p}(\Omega)\ {\sl if}\ k-{n\over p}<0;\cr
	&C^m(\bar{\Omega})\ {\sl if}\ 0\leq m\leq k-{n\over p};\cr
	&C^{m,\alpha}(\bar{\Omega})\ {\sl if}\ 0<m+\alpha<k-{n\over p}.
}\right.$$
Moreover, all embeddings are compact.

The following special cases are useful.
\smallskip
\item {(1)} When $k=1,p=2$, we have {\it Sobolev inequality} $W^{1,2}_0(\Omega)\hookrightarrow L^{2n\over n-2}(\Omega)$ for $n\geq 3$, or equivalently
$$\|f\|_{2n\over n-2}\leq C(\Omega)\|\nabla f\|_2.$$
\item {(2)} We have {\it Morrey's inequality}
$$W^{1,p}_0(\Omega)\hookrightarrow\left\{\eqalign{
	&L^{np\over n-p}(\Omega),\ n>p;\cr
	&C^0(\Omega),\ p>n;\cr
	&C^{0,\alpha},\ 0<\alpha<1-{n\over p}.
}\right.$$
\item {(3)} For $f\in W^{k,p}_0(\Omega)$ with $p>n$, we have $f\in C^{k-1}(\bar{\Omega})$.
\item {(4)} For $f\in W^{k,p}_0(\Omega)$ with $pk>n$, we have $f\in C^p(\bar{\Omega})$.
\medskip

\noindent{\bf Basic Question}.
In the embedding picture, why $k-{n\over p}$ appears?
\medskip

\noindent{\bf Philosophy}. Meaningful inequalities in geometry must be scaling invariant.
\medskip

\noindent{\bf Answer}. $k-{n\over p}$ comes from scaling argument.
\smallskip
\item {(1)} If $u\in C^{0,\alpha}(\Omega)$, then $|u(x)-u(y)|\leq C|x-y|^\alpha$ (H\"older).
Without loss of generality we can assume $\Omega=B_1(0)$, $y=0$.
Let $u_\lambda(x)=u(\lambda x)$, then
$$|u_\lambda(x)-u_\lambda(y)|=|u(\lambda x)-u(\lambda y)|\leq C|\lambda|^\alpha|x|^\alpha,$$
hence
$${|u_\lambda(x)-u_\lambda(y)|\over|\lambda|^\alpha}\leq C|x|^\alpha.$$
That is, H\"older function is of $C^{0,\alpha}$-regular.
\item {(2)} Assume $u\in L^2(B_1(0))$.
Let $u_\lambda(x)=u(\lambda x)$, then
$$\eqalign{
	\left(\int_{B_\lambda(0)}u_\lambda^2\right)^{1/2}&=\left(\int_{B_1(0)}u^2(\lambda x){d(\lambda x)\over\lambda^n}\right)^{1/2}\cr
	&=\lambda^{-n/2}\|u_\lambda\|_{L^2(B_\lambda(0))}.
}$$
That is, as domain scales $\lambda$ yields to the $L^2$-function scales like $\lambda^{-n/2}$.
Similarly, if $u\in L^p(B_1(0))$, $u$ is regular in the sense of $C^{-n/p}$.
\item {(3)} Assume $\nabla u\in L^2(B_1(0))$.
Let $u_\lambda(x)=u(\lambda x)$, then
$$\eqalign{
	\left(\int_{B_\lambda(0)}|\nabla u_\lambda|^2\right)^{1/2}&=\left(\int_{B_1(0)}{\lambda^2\over\lambda^n}|\nabla u(\lambda x)|^2d(\lambda x)\right)^{1/2}\cr
	&=\lambda^{1-n/2}\|\nabla u\|_{L^2(B_\lambda(0))}.
}$$
Similarly, if $u\in W^{k,p}(B_1(0))$, then $u$ scales like $\lambda^{k-n/p}$.
\medskip

\noindent{\bf Summary}. The three types of functions have regularity:
\smallskip
\item {(1)} H\"older function is $C^{0,\alpha}$-regular;
\item {(2)} $L^p$-function is $C^{-n/p}$-regular;
\item {(3)} $W^{k,p}$-function is $C^{k-n/p}$-regular.
\medskip

Now it's times to understand the Sobolev embedding.
$$W^{1,p}_0(\Omega)\hookrightarrow\left\{\eqalign{
	&L^{np\over n-p}(\Omega),\ p<n;\cr
	&C^{0,\alpha},\ 0<\alpha<1-{n\over p}.
}\right.$$
We have two situations:
\smallskip
\item {(1)} $p>n$, $u\in W^{1,p}_0(\Omega)$, then $u$ is $C^{1-n/p}$-regular at most;
\item {(2)} $p<n$, $1-{n\over p}<0$, then we expect $u\in L^q(\Omega)$ for some $q$.
Translate to inequality, we have 
$$\|u\|_{L^q(\Omega)}\leq C\|\nabla u\|_{L^p(\Omega)}.$$
By our ``philosophy'', the regularity on both sides of the inequality must equal.
The left side is $C^{-n/q}$-regular, and the right side is $C^{1-n/p}$-regular.
Thus
$$-{n\over q}=1-{n\over p},$$
we obtain $q={np\over n-p}$.

\endsection